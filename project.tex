% Options for packages loaded elsewhere
\PassOptionsToPackage{unicode}{hyperref}
\PassOptionsToPackage{hyphens}{url}
\PassOptionsToPackage{dvipsnames,svgnames,x11names}{xcolor}
%
\documentclass[
]{article}

\usepackage{amsmath,amssymb}
\usepackage{iftex}
\ifPDFTeX
  \usepackage[T1]{fontenc}
  \usepackage[utf8]{inputenc}
  \usepackage{textcomp} % provide euro and other symbols
\else % if luatex or xetex
  \usepackage{unicode-math}
  \defaultfontfeatures{Scale=MatchLowercase}
  \defaultfontfeatures[\rmfamily]{Ligatures=TeX,Scale=1}
\fi
\usepackage{lmodern}
\ifPDFTeX\else  
    % xetex/luatex font selection
\fi
% Use upquote if available, for straight quotes in verbatim environments
\IfFileExists{upquote.sty}{\usepackage{upquote}}{}
\IfFileExists{microtype.sty}{% use microtype if available
  \usepackage[]{microtype}
  \UseMicrotypeSet[protrusion]{basicmath} % disable protrusion for tt fonts
}{}
\makeatletter
\@ifundefined{KOMAClassName}{% if non-KOMA class
  \IfFileExists{parskip.sty}{%
    \usepackage{parskip}
  }{% else
    \setlength{\parindent}{0pt}
    \setlength{\parskip}{6pt plus 2pt minus 1pt}}
}{% if KOMA class
  \KOMAoptions{parskip=half}}
\makeatother
\usepackage{xcolor}
\setlength{\emergencystretch}{3em} % prevent overfull lines
\setcounter{secnumdepth}{5}
% Make \paragraph and \subparagraph free-standing
\makeatletter
\ifx\paragraph\undefined\else
  \let\oldparagraph\paragraph
  \renewcommand{\paragraph}{
    \@ifstar
      \xxxParagraphStar
      \xxxParagraphNoStar
  }
  \newcommand{\xxxParagraphStar}[1]{\oldparagraph*{#1}\mbox{}}
  \newcommand{\xxxParagraphNoStar}[1]{\oldparagraph{#1}\mbox{}}
\fi
\ifx\subparagraph\undefined\else
  \let\oldsubparagraph\subparagraph
  \renewcommand{\subparagraph}{
    \@ifstar
      \xxxSubParagraphStar
      \xxxSubParagraphNoStar
  }
  \newcommand{\xxxSubParagraphStar}[1]{\oldsubparagraph*{#1}\mbox{}}
  \newcommand{\xxxSubParagraphNoStar}[1]{\oldsubparagraph{#1}\mbox{}}
\fi
\makeatother

\usepackage{color}
\usepackage{fancyvrb}
\newcommand{\VerbBar}{|}
\newcommand{\VERB}{\Verb[commandchars=\\\{\}]}
\DefineVerbatimEnvironment{Highlighting}{Verbatim}{commandchars=\\\{\}}
% Add ',fontsize=\small' for more characters per line
\usepackage{framed}
\definecolor{shadecolor}{RGB}{241,243,245}
\newenvironment{Shaded}{\begin{snugshade}}{\end{snugshade}}
\newcommand{\AlertTok}[1]{\textcolor[rgb]{0.68,0.00,0.00}{#1}}
\newcommand{\AnnotationTok}[1]{\textcolor[rgb]{0.37,0.37,0.37}{#1}}
\newcommand{\AttributeTok}[1]{\textcolor[rgb]{0.40,0.45,0.13}{#1}}
\newcommand{\BaseNTok}[1]{\textcolor[rgb]{0.68,0.00,0.00}{#1}}
\newcommand{\BuiltInTok}[1]{\textcolor[rgb]{0.00,0.23,0.31}{#1}}
\newcommand{\CharTok}[1]{\textcolor[rgb]{0.13,0.47,0.30}{#1}}
\newcommand{\CommentTok}[1]{\textcolor[rgb]{0.37,0.37,0.37}{#1}}
\newcommand{\CommentVarTok}[1]{\textcolor[rgb]{0.37,0.37,0.37}{\textit{#1}}}
\newcommand{\ConstantTok}[1]{\textcolor[rgb]{0.56,0.35,0.01}{#1}}
\newcommand{\ControlFlowTok}[1]{\textcolor[rgb]{0.00,0.23,0.31}{\textbf{#1}}}
\newcommand{\DataTypeTok}[1]{\textcolor[rgb]{0.68,0.00,0.00}{#1}}
\newcommand{\DecValTok}[1]{\textcolor[rgb]{0.68,0.00,0.00}{#1}}
\newcommand{\DocumentationTok}[1]{\textcolor[rgb]{0.37,0.37,0.37}{\textit{#1}}}
\newcommand{\ErrorTok}[1]{\textcolor[rgb]{0.68,0.00,0.00}{#1}}
\newcommand{\ExtensionTok}[1]{\textcolor[rgb]{0.00,0.23,0.31}{#1}}
\newcommand{\FloatTok}[1]{\textcolor[rgb]{0.68,0.00,0.00}{#1}}
\newcommand{\FunctionTok}[1]{\textcolor[rgb]{0.28,0.35,0.67}{#1}}
\newcommand{\ImportTok}[1]{\textcolor[rgb]{0.00,0.46,0.62}{#1}}
\newcommand{\InformationTok}[1]{\textcolor[rgb]{0.37,0.37,0.37}{#1}}
\newcommand{\KeywordTok}[1]{\textcolor[rgb]{0.00,0.23,0.31}{\textbf{#1}}}
\newcommand{\NormalTok}[1]{\textcolor[rgb]{0.00,0.23,0.31}{#1}}
\newcommand{\OperatorTok}[1]{\textcolor[rgb]{0.37,0.37,0.37}{#1}}
\newcommand{\OtherTok}[1]{\textcolor[rgb]{0.00,0.23,0.31}{#1}}
\newcommand{\PreprocessorTok}[1]{\textcolor[rgb]{0.68,0.00,0.00}{#1}}
\newcommand{\RegionMarkerTok}[1]{\textcolor[rgb]{0.00,0.23,0.31}{#1}}
\newcommand{\SpecialCharTok}[1]{\textcolor[rgb]{0.37,0.37,0.37}{#1}}
\newcommand{\SpecialStringTok}[1]{\textcolor[rgb]{0.13,0.47,0.30}{#1}}
\newcommand{\StringTok}[1]{\textcolor[rgb]{0.13,0.47,0.30}{#1}}
\newcommand{\VariableTok}[1]{\textcolor[rgb]{0.07,0.07,0.07}{#1}}
\newcommand{\VerbatimStringTok}[1]{\textcolor[rgb]{0.13,0.47,0.30}{#1}}
\newcommand{\WarningTok}[1]{\textcolor[rgb]{0.37,0.37,0.37}{\textit{#1}}}

\providecommand{\tightlist}{%
  \setlength{\itemsep}{0pt}\setlength{\parskip}{0pt}}\usepackage{longtable,booktabs,array}
\usepackage{calc} % for calculating minipage widths
% Correct order of tables after \paragraph or \subparagraph
\usepackage{etoolbox}
\makeatletter
\patchcmd\longtable{\par}{\if@noskipsec\mbox{}\fi\par}{}{}
\makeatother
% Allow footnotes in longtable head/foot
\IfFileExists{footnotehyper.sty}{\usepackage{footnotehyper}}{\usepackage{footnote}}
\makesavenoteenv{longtable}
\usepackage{graphicx}
\makeatletter
\newsavebox\pandoc@box
\newcommand*\pandocbounded[1]{% scales image to fit in text height/width
  \sbox\pandoc@box{#1}%
  \Gscale@div\@tempa{\textheight}{\dimexpr\ht\pandoc@box+\dp\pandoc@box\relax}%
  \Gscale@div\@tempb{\linewidth}{\wd\pandoc@box}%
  \ifdim\@tempb\p@<\@tempa\p@\let\@tempa\@tempb\fi% select the smaller of both
  \ifdim\@tempa\p@<\p@\scalebox{\@tempa}{\usebox\pandoc@box}%
  \else\usebox{\pandoc@box}%
  \fi%
}
% Set default figure placement to htbp
\def\fps@figure{htbp}
\makeatother

\usepackage{booktabs}
\usepackage{longtable}
\usepackage{array}
\usepackage{multirow}
\usepackage{wrapfig}
\usepackage{float}
\usepackage{colortbl}
\usepackage{pdflscape}
\usepackage{tabu}
\usepackage{threeparttable}
\usepackage{threeparttablex}
\usepackage[normalem]{ulem}
\usepackage{makecell}
\usepackage{xcolor}
\makeatletter
\@ifpackageloaded{caption}{}{\usepackage{caption}}
\AtBeginDocument{%
\ifdefined\contentsname
  \renewcommand*\contentsname{Table of contents}
\else
  \newcommand\contentsname{Table of contents}
\fi
\ifdefined\listfigurename
  \renewcommand*\listfigurename{List of Figures}
\else
  \newcommand\listfigurename{List of Figures}
\fi
\ifdefined\listtablename
  \renewcommand*\listtablename{List of Tables}
\else
  \newcommand\listtablename{List of Tables}
\fi
\ifdefined\figurename
  \renewcommand*\figurename{Figure}
\else
  \newcommand\figurename{Figure}
\fi
\ifdefined\tablename
  \renewcommand*\tablename{Table}
\else
  \newcommand\tablename{Table}
\fi
}
\@ifpackageloaded{float}{}{\usepackage{float}}
\floatstyle{ruled}
\@ifundefined{c@chapter}{\newfloat{codelisting}{h}{lop}}{\newfloat{codelisting}{h}{lop}[chapter]}
\floatname{codelisting}{Listing}
\newcommand*\listoflistings{\listof{codelisting}{List of Listings}}
\makeatother
\makeatletter
\makeatother
\makeatletter
\@ifpackageloaded{caption}{}{\usepackage{caption}}
\@ifpackageloaded{subcaption}{}{\usepackage{subcaption}}
\makeatother

\usepackage{bookmark}

\IfFileExists{xurl.sty}{\usepackage{xurl}}{} % add URL line breaks if available
\urlstyle{same} % disable monospaced font for URLs
\hypersetup{
  pdftitle={It's Not Rocket Science, It's Electricity Market},
  colorlinks=true,
  linkcolor={blue},
  filecolor={Maroon},
  citecolor={Blue},
  urlcolor={Blue},
  pdfcreator={LaTeX via pandoc}}


\title{It's Not Rocket Science, It's Electricity Market}
\author{}
\date{}

\begin{document}
\maketitle

\renewcommand*\contentsname{Table of contents}
{
\hypersetup{linkcolor=}
\setcounter{tocdepth}{3}
\tableofcontents
}

Welcome to my EMU660 project page.

\section{1. Project Overview and
Scope}\label{project-overview-and-scope}

Electricity has increasingly become a tradable commodity on global and
Turkish stock exchanges, subject to specific regulations and
limitations. In a liberalized market, it is uniquely characterized by a
third dimension---time---alongside price and volume. In Turkey,
electricity trading takes place across multiple market platforms, all
overseen and regulated by the Energy Exchange Istanbul (EXIST). This
project aims to analyze the formation of electricity prices and
investigate the impact of total electricity demand as well as
electricity generation from various sources on price formation.
Specifically, the daily impact of renewable energy generation on
electricity prices will be examined, while the influence of natural gas
prices on the monthly average electricity price will also be explored.
Electricity prices will be forecasted using multiple linear regression
models at both daily and monthly resolutions, and the results will be
evaluated accordingly.

\section{2. Data}\label{data}

This project will utilize three main data sources related to the
electricity market or influence it. EPİAŞ (Energy Exchange Istanbul),
TEİAŞ (Turkish Electricity Transmission Corporation), and BOTAŞ
(Petroleum Pipeline Corporation) are public institutions in Turkey that
act as decision-makers and regulators in the electricity market. The
open-access data provided by these institutions will be used throughout
the analysis and forecasting processes of the project. The whole data
used in this project starts in first hour of 2023 and end at the end of
2024.

\subsection{2.1 Data Source}\label{data-source}

This project will utilize three main data sources related to the
electricity market or influence it. EPİAŞ (Energy Exchange Istanbul),
TEİAŞ (Turkish Electricity Transmission Corporation), and BOTAŞ
(Petroleum Pipeline Corporation) are public institutions in Turkey that
act as decision-makers and regulators in the electricity market. The
open-access data provided by these institutions will be used throughout
the analysis and forecasting processes of the project. Data sources can
be reached by clicking links below.

\href{https://seffaflik.epias.com.tr/home}{EPİAŞ}

\href{https://www.teias.gov.tr/sektor-raporlari}{TEİAŞ}

\href{https://www.botas.gov.tr/Sayfa/satis-fiyat-tarifesi/439}{BOTAŞ}

\subsection{2.2 General Information About
Data}\label{general-information-about-data}

Data related to the Day-Ahead Market will be obtained from EPİAŞ. The
data sourced from EPİAŞ can be categorized under three main headings.

\begin{enumerate}
\def\labelenumi{\arabic{enumi})}
\item
  FDDP (Final Daily Production Program): This data is provided at an
  hourly resolution for 12 different types of energy sources. It
  includes the planned generation amounts for the following day
  submitted by power plants operating under each source category. Every
  day, power plants enter their generation schedules into the system by
  4 PM, and EPİAŞ collects and publishes this data aggregated by source
  type.
\item
  Real-Time Consumption: This data represents the total amount of
  electricity consumed across Turkey. It is provided on an hourly basis
  and can be referred to as the total electricity demand.
\item
  MCP (Market Clearing Price): This data refers to the electricity price
  determined for each hour in the Day-Ahead Market, formed by matching
  supply and demand for the traded electricity.
\end{enumerate}

Natural gas tariff data has been sourced from BOTAŞ. The prices of
natural gas used for electricity generation are determined by BOTAŞ.
Additionally, water inflow data to the main basin dams, provided by
TEİAŞ, may be used if deemed necessary.

\subsection{2.3 Reason of Choice}\label{reason-of-choice}

The electricity market consists of various sub-markets. Making accurate
price forecasts for short-term and long-term electricity sales can
create significant added value. Especially in long-term purchase or sale
agreements, forecasting electricity prices can facilitate more
profitable commercial deals while minimizing risk. For instance, the
analyses and models developed in this project can help establish a
relationship between renewable energy generation and electricity prices
over specific periods. These forecasts can then be used to assess buy
and sell offers in the market for future periods, enabling more informed
and strategic positioning.

\subsection{2.4 Preprocessing}\label{preprocessing}

In the preprocessing stage, the data stored in Excel files was converted
into RData format.

\begin{Shaded}
\begin{Highlighting}[]
\FunctionTok{library}\NormalTok{(readxl)}

\NormalTok{epias\_data }\OtherTok{\textless{}{-}} \FunctionTok{read\_excel}\NormalTok{(}\StringTok{"epias\_data.xlsx"}\NormalTok{)}

\NormalTok{botas\_data }\OtherTok{\textless{}{-}} \FunctionTok{read\_excel}\NormalTok{(}\StringTok{"botas\_data.xlsx"}\NormalTok{)}

\FunctionTok{save}\NormalTok{(epias\_data, botas\_data, }\AttributeTok{file =} \StringTok{"electricity.RData"}\NormalTok{)}

\FunctionTok{head}\NormalTok{(epias\_data)}
\end{Highlighting}
\end{Shaded}

\begin{verbatim}
# A tibble: 6 x 18
  date                hour   total naturalgas  wind lignite darkcoal
  <dttm>              <chr>  <dbl>      <dbl> <dbl>   <dbl>    <dbl>
1 2023-01-01 00:00:00 00:00 25945.      3636.  988.   4646.      136
2 2023-01-01 01:00:00 01:00 24494.      2939. 1056.   4646.      136
3 2023-01-01 02:00:00 02:00 22631.      2495. 1123.   4694.      136
4 2023-01-01 03:00:00 03:00 22022.      2667. 1227.   4878.      136
5 2023-01-01 04:00:00 04:00 21404.      2334. 1308.   4910.      136
6 2023-01-01 05:00:00 05:00 21586.      2177. 1406.   5016.      136
# i 11 more variables: importedcoal <dbl>, fueloil <dbl>, geothermal <dbl>,
#   dam <dbl>, naphta <dbl>, biomass <dbl>, runofriver <dbl>, other <dbl>,
#   demand <dbl>, solar <dbl>, price <dbl>
\end{verbatim}

\begin{Shaded}
\begin{Highlighting}[]
\FunctionTok{head}\NormalTok{(botas\_data)}
\end{Highlighting}
\end{Shaded}

\begin{verbatim}
# A tibble: 6 x 3
   year month natgasprice
  <dbl> <dbl>       <dbl>
1  2023     1       18000
2  2023     2       15000
3  2023     3       12000
4  2023     4       10000
5  2023     5       10000
6  2023     6       10000
\end{verbatim}

The natural gas price data obtained from BOTAŞ was available on a
monthly basis. These monthly values were integrated into the hourly
dataset.

\begin{Shaded}
\begin{Highlighting}[]
\FunctionTok{load}\NormalTok{(}\StringTok{"electricity.RData"}\NormalTok{)}
\FunctionTok{library}\NormalTok{(dplyr)}
\end{Highlighting}
\end{Shaded}

\begin{verbatim}

Attaching package: 'dplyr'
\end{verbatim}

\begin{verbatim}
The following objects are masked from 'package:stats':

    filter, lag
\end{verbatim}

\begin{verbatim}
The following objects are masked from 'package:base':

    intersect, setdiff, setequal, union
\end{verbatim}

\begin{Shaded}
\begin{Highlighting}[]
\NormalTok{epias\_data}\SpecialCharTok{$}\NormalTok{date }\OtherTok{\textless{}{-}} \FunctionTok{as.Date}\NormalTok{(epias\_data}\SpecialCharTok{$}\NormalTok{date)}

\NormalTok{epias\_data}\SpecialCharTok{$}\NormalTok{year }\OtherTok{\textless{}{-}} \FunctionTok{format}\NormalTok{(epias\_data}\SpecialCharTok{$}\NormalTok{date, }\StringTok{"\%Y"}\NormalTok{)}
\NormalTok{epias\_data}\SpecialCharTok{$}\NormalTok{month }\OtherTok{\textless{}{-}} \FunctionTok{format}\NormalTok{(epias\_data}\SpecialCharTok{$}\NormalTok{date, }\StringTok{"\%m"}\NormalTok{)}

\NormalTok{botas\_data}\SpecialCharTok{$}\NormalTok{month }\OtherTok{\textless{}{-}} \FunctionTok{sprintf}\NormalTok{(}\StringTok{"\%02d"}\NormalTok{, botas\_data}\SpecialCharTok{$}\NormalTok{month)}

\NormalTok{botas\_data}\SpecialCharTok{$}\NormalTok{year }\OtherTok{\textless{}{-}} \FunctionTok{as.character}\NormalTok{(botas\_data}\SpecialCharTok{$}\NormalTok{year)}

\NormalTok{epias\_merged }\OtherTok{\textless{}{-}} \FunctionTok{left\_join}\NormalTok{(epias\_data, botas\_data, }\AttributeTok{by =} \FunctionTok{c}\NormalTok{(}\StringTok{"year"}\NormalTok{, }\StringTok{"month"}\NormalTok{))}

\FunctionTok{head}\NormalTok{(epias\_merged)}
\end{Highlighting}
\end{Shaded}

\begin{verbatim}
# A tibble: 6 x 21
  date       hour   total naturalgas  wind lignite darkcoal importedcoal fueloil
  <date>     <chr>  <dbl>      <dbl> <dbl>   <dbl>    <dbl>        <dbl>   <dbl>
1 2023-01-01 00:00 25945.      3636.  988.   4646.      136        8556.    16.5
2 2023-01-01 01:00 24494.      2939. 1056.   4646.      136        8556.    16.5
3 2023-01-01 02:00 22631.      2495. 1123.   4694.      136        8554.    16.5
4 2023-01-01 03:00 22022.      2667. 1227.   4878.      136        8372.    16.5
5 2023-01-01 04:00 21404.      2334. 1308.   4910.      136        8373.    16.5
6 2023-01-01 05:00 21586.      2177. 1406.   5016.      136        8373.    15.5
# i 12 more variables: geothermal <dbl>, dam <dbl>, naphta <dbl>,
#   biomass <dbl>, runofriver <dbl>, other <dbl>, demand <dbl>, solar <dbl>,
#   price <dbl>, year <chr>, month <chr>, natgasprice <dbl>
\end{verbatim}

\begin{Shaded}
\begin{Highlighting}[]
\FunctionTok{save}\NormalTok{(epias\_merged, }\AttributeTok{file =} \StringTok{"electricity\_merged.RData"}\NormalTok{)}

\NormalTok{epias\_merged }\OtherTok{\textless{}{-}}\NormalTok{ epias\_merged }\SpecialCharTok{\%\textgreater{}\%}
  \FunctionTok{select}\NormalTok{(date, year, month, }\FunctionTok{everything}\NormalTok{(), }\SpecialCharTok{{-}}\NormalTok{year, }\SpecialCharTok{{-}}\NormalTok{month)}

\NormalTok{epias\_merged }\OtherTok{\textless{}{-}}\NormalTok{ epias\_merged }\SpecialCharTok{\%\textgreater{}\%}
  \FunctionTok{select}\NormalTok{(}\SpecialCharTok{{-}}\NormalTok{price, }\FunctionTok{everything}\NormalTok{(), price)}

\FunctionTok{save}\NormalTok{(epias\_merged, }\AttributeTok{file =} \StringTok{"electricity\_merged.RData"}\NormalTok{)}
\end{Highlighting}
\end{Shaded}

The hourly data was converted into daily averages to enable analysis at
a daily resolution.

\begin{Shaded}
\begin{Highlighting}[]
\FunctionTok{library}\NormalTok{(dplyr)}

\NormalTok{epias\_daily }\OtherTok{\textless{}{-}}\NormalTok{ epias\_merged }\SpecialCharTok{\%\textgreater{}\%}
  \FunctionTok{group\_by}\NormalTok{(date) }\SpecialCharTok{\%\textgreater{}\%}
  \FunctionTok{summarise}\NormalTok{(}\FunctionTok{across}\NormalTok{(}\FunctionTok{where}\NormalTok{(is.numeric), mean, }\AttributeTok{na.rm =} \ConstantTok{TRUE}\NormalTok{))}
\end{Highlighting}
\end{Shaded}

\begin{verbatim}
Warning: There was 1 warning in `summarise()`.
i In argument: `across(where(is.numeric), mean, na.rm = TRUE)`.
i In group 1: `date = 2023-01-01`.
Caused by warning:
! The `...` argument of `across()` is deprecated as of dplyr 1.1.0.
Supply arguments directly to `.fns` through an anonymous function instead.

  # Previously
  across(a:b, mean, na.rm = TRUE)

  # Now
  across(a:b, \(x) mean(x, na.rm = TRUE))
\end{verbatim}

Finally, a feature aggregation process was carried out to prepare the
data for forecasting. Generation sources with similar characteristics
were grouped under common categories. Fueloil, naphta, lignite, and hard
coal were combined under the label cheap\_thermal. Wind, run-of-river,
biomass, and geothermal sources were grouped under renewables.

Solar and hydro (dam) generation were excluded from the renewables
group, as they exhibit distinct production characteristics. The category
other was disregarded due to its low share and lack of detailed
classification.

\begin{Shaded}
\begin{Highlighting}[]
\FunctionTok{library}\NormalTok{(dplyr)}

\NormalTok{epias\_simplified\_daily }\OtherTok{\textless{}{-}}\NormalTok{ epias\_daily }\SpecialCharTok{\%\textgreater{}\%}
  \FunctionTok{mutate}\NormalTok{(}
    \AttributeTok{cheap\_thermal =}\NormalTok{ fueloil }\SpecialCharTok{+}\NormalTok{ naphta }\SpecialCharTok{+}\NormalTok{ lignite }\SpecialCharTok{+}\NormalTok{ darkcoal,}
    \AttributeTok{renewables =}\NormalTok{ wind }\SpecialCharTok{+}\NormalTok{ runofriver }\SpecialCharTok{+}\NormalTok{ biomass }\SpecialCharTok{+}\NormalTok{ geothermal}
\NormalTok{  ) }\SpecialCharTok{\%\textgreater{}\%}
  \FunctionTok{select}\NormalTok{(}
\NormalTok{    date, cheap\_thermal, renewables,}
\NormalTok{    importedcoal, naturalgas, solar, dam, demand, natgasprice, price}
\NormalTok{  )}
\FunctionTok{save}\NormalTok{(epias\_simplified\_daily, }\AttributeTok{file =} \StringTok{"epias\_simplified\_daily.RData"}\NormalTok{)}

\FunctionTok{head}\NormalTok{(epias\_simplified\_daily)}
\end{Highlighting}
\end{Shaded}

\begin{verbatim}
# A tibble: 6 x 10
  date       cheap_thermal renewables importedcoal naturalgas solar   dam demand
  <date>             <dbl>      <dbl>        <dbl>      <dbl> <dbl> <dbl>  <dbl>
1 2023-01-01         4972.      4375.        8332.      3210. 1491. 3100. 28743.
2 2023-01-02         5128.      3919.        9062.      8888. 1441. 2541. 35772.
3 2023-01-03         5016.      4079.        9328.     11235. 1417. 2531. 37497.
4 2023-01-04         4834.      5007.        8810.     10960. 1345. 2213. 38064.
5 2023-01-05         4849.      4986.        8435.     11628. 1207. 2367. 37877.
6 2023-01-06         4950.      4648.        8633.     11608. 1126. 2334. 37821.
# i 2 more variables: natgasprice <dbl>, price <dbl>
\end{verbatim}

\section{3. Analysis}\label{analysis}

\subsection{3.1 Why It's Important to Create a
Forecast?}\label{why-its-important-to-create-a-forecast}

Electricity price forecasting plays a critical role in the power market.
Accurate forecasts are essential not only for making commercial
decisions but also for managing financial processes. Cash flow
management is a key factor for maintaining an active and balanced
presence in the market. Portfolios with constant inflows and outflows
must manage their commercial balance while simultaneously overseeing
their cash flows. Daily electricity price forecasting can provide
significant advantages in both commercial and financial foresight.

\subsection{3.2 How Electricity Prices are
Determined?}\label{how-electricity-prices-are-determined}

Electricity prices are determined through a system known as the merit
order. Similar to basic economic pricing, supply and demand are matched
for each hour, and the price is set at the point where supply meets
demand. However, unlike other markets, electricity is a fundamental
need---so demand does not respond to price, but instead determines it.

For any given hour, demand is met starting from the cheapest suppliers,
moving up to the more expensive ones. The price of electricity is then
set based on the bid of the most expensive accepted supplier.
Electricity generation resources can generally be ranked from cheapest
to most expensive as follows: renewables, hydro, nuclear, domestic coal,
imported coal, and natural gas.

\pandocbounded{\includegraphics[keepaspectratio]{images/Merit-Order-Effect-Chart_Squeaky-01.png}}

\subsection{\texorpdfstring{3.3 \textbf{Exploratory Data Analysis \&
Trend
Analysis}}{3.3 Exploratory Data Analysis \& Trend Analysis}}\label{exploratory-data-analysis-trend-analysis}

Electricity prices can be highly volatile even on the same type of day.
The chart below shows the prices for two different Mondays in November
2024

\begin{Shaded}
\begin{Highlighting}[]
\FunctionTok{library}\NormalTok{(dplyr)}
\FunctionTok{library}\NormalTok{(ggplot2)}

\CommentTok{\# Veri seçimi}
\NormalTok{price\_data }\OtherTok{\textless{}{-}}\NormalTok{ epias\_merged }\SpecialCharTok{\%\textgreater{}\%}
  \FunctionTok{filter}\NormalTok{(date }\SpecialCharTok{\%in\%} \FunctionTok{as.Date}\NormalTok{(}\FunctionTok{c}\NormalTok{(}\StringTok{"2024{-}11{-}04"}\NormalTok{, }\StringTok{"2024{-}11{-}18"}\NormalTok{))) }\SpecialCharTok{\%\textgreater{}\%}
  \FunctionTok{mutate}\NormalTok{(}\AttributeTok{hour\_numeric =} \FunctionTok{as.numeric}\NormalTok{(}\FunctionTok{substr}\NormalTok{(hour, }\DecValTok{1}\NormalTok{, }\DecValTok{2}\NormalTok{))) }\SpecialCharTok{\%\textgreater{}\%}
  \FunctionTok{select}\NormalTok{(date, hour\_numeric, price)}

\CommentTok{\# Grafik}
\FunctionTok{ggplot}\NormalTok{(price\_data, }\FunctionTok{aes}\NormalTok{(}\AttributeTok{x =}\NormalTok{ hour\_numeric, }\AttributeTok{y =}\NormalTok{ price, }\AttributeTok{color =} \FunctionTok{as.factor}\NormalTok{(date), }\AttributeTok{group =}\NormalTok{ date)) }\SpecialCharTok{+}
  \FunctionTok{geom\_line}\NormalTok{(}\AttributeTok{size =} \FloatTok{1.2}\NormalTok{) }\SpecialCharTok{+}
  \FunctionTok{scale\_color\_manual}\NormalTok{(}
    \AttributeTok{values =} \FunctionTok{c}\NormalTok{(}\StringTok{"2024{-}11{-}04"} \OtherTok{=} \StringTok{"red"}\NormalTok{, }\StringTok{"2024{-}11{-}18"} \OtherTok{=} \StringTok{"blue"}\NormalTok{),}
    \AttributeTok{labels =} \FunctionTok{c}\NormalTok{(}\StringTok{"4 November 2024"}\NormalTok{, }\StringTok{"18 November 2024"}\NormalTok{)}
\NormalTok{  ) }\SpecialCharTok{+}
  \FunctionTok{scale\_x\_continuous}\NormalTok{(}
    \AttributeTok{breaks =} \DecValTok{0}\SpecialCharTok{:}\DecValTok{23}\NormalTok{,}
    \AttributeTok{labels =} \FunctionTok{sprintf}\NormalTok{(}\StringTok{"\%02d:00"}\NormalTok{, }\DecValTok{0}\SpecialCharTok{:}\DecValTok{23}\NormalTok{)}
\NormalTok{  ) }\SpecialCharTok{+}
  \FunctionTok{labs}\NormalTok{(}
    \AttributeTok{title =} \StringTok{"Hourly Electricity Price – 4 \& 18 November 2024"}\NormalTok{,}
    \AttributeTok{x =} \StringTok{"Hour of Day"}\NormalTok{,}
    \AttributeTok{y =} \StringTok{"Price (TL/MWh)"}\NormalTok{,}
    \AttributeTok{color =} \StringTok{"Date"}
\NormalTok{  ) }\SpecialCharTok{+}
  \FunctionTok{theme\_minimal}\NormalTok{() }\SpecialCharTok{+}
  \FunctionTok{theme}\NormalTok{(}
    \AttributeTok{plot.title =} \FunctionTok{element\_text}\NormalTok{(}\AttributeTok{hjust =} \FloatTok{0.5}\NormalTok{, }\AttributeTok{face =} \StringTok{"bold"}\NormalTok{),}
    \AttributeTok{axis.text.x =} \FunctionTok{element\_text}\NormalTok{(}\AttributeTok{angle =} \DecValTok{45}\NormalTok{, }\AttributeTok{hjust =} \DecValTok{1}\NormalTok{)}
\NormalTok{  )}
\end{Highlighting}
\end{Shaded}

\begin{verbatim}
Warning: Using `size` aesthetic for lines was deprecated in ggplot2 3.4.0.
i Please use `linewidth` instead.
\end{verbatim}

\pandocbounded{\includegraphics[keepaspectratio]{project_files/figure-pdf/unnamed-chunk-5-1.pdf}}

Electricity prices exhibit high volatility across different hours of the
day. This volatility is clearly visible in the chart below, with midday
hours standing out as particularly volatile.

\begin{Shaded}
\begin{Highlighting}[]
\FunctionTok{library}\NormalTok{(dplyr)}
\FunctionTok{library}\NormalTok{(ggplot2)}
\FunctionTok{library}\NormalTok{(lubridate)}
\end{Highlighting}
\end{Shaded}

\begin{verbatim}

Attaching package: 'lubridate'
\end{verbatim}

\begin{verbatim}
The following objects are masked from 'package:base':

    date, intersect, setdiff, union
\end{verbatim}

\begin{Shaded}
\begin{Highlighting}[]
\CommentTok{\# 1. Saat sütunu sayıya çevrilir}
\NormalTok{epias\_merged }\OtherTok{\textless{}{-}}\NormalTok{ epias\_merged }\SpecialCharTok{\%\textgreater{}\%}
  \FunctionTok{mutate}\NormalTok{(}\AttributeTok{hour\_numeric =} \FunctionTok{as.numeric}\NormalTok{(}\FunctionTok{substr}\NormalTok{(hour, }\DecValTok{1}\NormalTok{, }\DecValTok{2}\NormalTok{)))}

\CommentTok{\# 2. Sadece 2024 yılı alınır}
\NormalTok{price\_2024 }\OtherTok{\textless{}{-}}\NormalTok{ epias\_merged }\SpecialCharTok{\%\textgreater{}\%}
  \FunctionTok{filter}\NormalTok{(}\FunctionTok{year}\NormalTok{(date) }\SpecialCharTok{==} \DecValTok{2024}\NormalTok{) }\SpecialCharTok{\%\textgreater{}\%}
  \FunctionTok{select}\NormalTok{(hour\_numeric, price)}

\CommentTok{\# 3. Violin plot: her saat için fiyat yoğunluğu}
\FunctionTok{ggplot}\NormalTok{(price\_2024, }\FunctionTok{aes}\NormalTok{(}\AttributeTok{x =} \FunctionTok{factor}\NormalTok{(hour\_numeric), }\AttributeTok{y =}\NormalTok{ price, }\AttributeTok{fill =} \FunctionTok{factor}\NormalTok{(hour\_numeric))) }\SpecialCharTok{+}
  \FunctionTok{geom\_violin}\NormalTok{(}\AttributeTok{scale =} \StringTok{"width"}\NormalTok{, }\AttributeTok{adjust =} \FloatTok{1.2}\NormalTok{, }\AttributeTok{alpha =} \FloatTok{0.8}\NormalTok{, }\AttributeTok{color =} \ConstantTok{NA}\NormalTok{) }\SpecialCharTok{+}
  \FunctionTok{scale\_fill\_viridis\_d}\NormalTok{(}\AttributeTok{option =} \StringTok{"C"}\NormalTok{, }\AttributeTok{begin =} \FloatTok{0.1}\NormalTok{, }\AttributeTok{end =} \FloatTok{0.9}\NormalTok{) }\SpecialCharTok{+}
  \FunctionTok{coord\_cartesian}\NormalTok{(}\AttributeTok{ylim =} \FunctionTok{c}\NormalTok{(}\DecValTok{0}\NormalTok{, }\DecValTok{3400}\NormalTok{)) }\SpecialCharTok{+}  \CommentTok{\# Görsel netlik için üst sınır}
  \FunctionTok{labs}\NormalTok{(}
    \AttributeTok{title =} \StringTok{"Hourly Electricity Price Distribution – 2024"}\NormalTok{,}
    \AttributeTok{x =} \StringTok{"Hour of Day"}\NormalTok{,}
    \AttributeTok{y =} \StringTok{"Electricity Price (TL/MWh)"}\NormalTok{,}
    \AttributeTok{fill =} \StringTok{"Hour"}
\NormalTok{  ) }\SpecialCharTok{+}
  \FunctionTok{theme\_minimal}\NormalTok{() }\SpecialCharTok{+}
  \FunctionTok{theme}\NormalTok{(}
    \AttributeTok{plot.title =} \FunctionTok{element\_text}\NormalTok{(}\AttributeTok{hjust =} \FloatTok{0.5}\NormalTok{, }\AttributeTok{face =} \StringTok{"bold"}\NormalTok{, }\AttributeTok{size =} \DecValTok{14}\NormalTok{),}
    \AttributeTok{plot.subtitle =} \FunctionTok{element\_text}\NormalTok{(}\AttributeTok{hjust =} \FloatTok{0.5}\NormalTok{),}
    \AttributeTok{legend.position =} \StringTok{"none"}
\NormalTok{  )}
\end{Highlighting}
\end{Shaded}

\pandocbounded{\includegraphics[keepaspectratio]{project_files/figure-pdf/unnamed-chunk-6-1.pdf}}

To understand the electricity market, it is essential to first examine
the factors that influence it. Among these, the most critical elements
are generation and consumption data.

\begin{Shaded}
\begin{Highlighting}[]
\FunctionTok{library}\NormalTok{(dplyr)}
\FunctionTok{library}\NormalTok{(tidyr)}
\FunctionTok{library}\NormalTok{(ggplot2)}
\FunctionTok{library}\NormalTok{(lubridate)}

\NormalTok{prod\_data }\OtherTok{\textless{}{-}}\NormalTok{ epias\_merged }\SpecialCharTok{\%\textgreater{}\%}
  \FunctionTok{filter}\NormalTok{(date }\SpecialCharTok{==} \FunctionTok{as.Date}\NormalTok{(}\StringTok{"2024{-}06{-}12"}\NormalTok{)) }\SpecialCharTok{\%\textgreater{}\%}
  \FunctionTok{select}\NormalTok{(hour, solar, wind, runofriver, dam, geothermal, biomass, naturalgas,}
\NormalTok{         fueloil, naphta, lignite, darkcoal, importedcoal, other)}

\NormalTok{prod\_long }\OtherTok{\textless{}{-}}\NormalTok{ prod\_data }\SpecialCharTok{\%\textgreater{}\%}
  \FunctionTok{pivot\_longer}\NormalTok{(}
    \AttributeTok{cols =} \SpecialCharTok{{-}}\NormalTok{hour,}
    \AttributeTok{names\_to =} \StringTok{"source"}\NormalTok{,}
    \AttributeTok{values\_to =} \StringTok{"generation"}
\NormalTok{  )}

\NormalTok{source\_colors }\OtherTok{\textless{}{-}} \FunctionTok{c}\NormalTok{(}
  \AttributeTok{solar =} \StringTok{"gold"}\NormalTok{,}
  \AttributeTok{wind =} \StringTok{"forestgreen"}\NormalTok{,}
  \AttributeTok{runofriver =} \StringTok{"dodgerblue"}\NormalTok{,}
  \AttributeTok{dam =} \StringTok{"steelblue"}\NormalTok{,}
  \AttributeTok{geothermal =} \StringTok{"sienna"}\NormalTok{,}
  \AttributeTok{biomass =} \StringTok{"darkolivegreen"}\NormalTok{,}
  \AttributeTok{naturalgas =} \StringTok{"firebrick"}\NormalTok{,}
  \AttributeTok{fueloil =} \StringTok{"darkred"}\NormalTok{,}
  \AttributeTok{naphta =} \StringTok{"orangered"}\NormalTok{,}
  \AttributeTok{lignite =} \StringTok{"black"}\NormalTok{,}
  \AttributeTok{darkcoal =} \StringTok{"black"}\NormalTok{,}
  \AttributeTok{importedcoal =} \StringTok{"dimgray"}\NormalTok{,}
  \AttributeTok{other =} \StringTok{"purple"}
\NormalTok{)}

\FunctionTok{ggplot}\NormalTok{(prod\_long, }\FunctionTok{aes}\NormalTok{(}\AttributeTok{x =}\NormalTok{ hour, }\AttributeTok{y =}\NormalTok{ generation, }\AttributeTok{fill =}\NormalTok{ source)) }\SpecialCharTok{+}
  \FunctionTok{geom\_bar}\NormalTok{(}\AttributeTok{stat =} \StringTok{"identity"}\NormalTok{) }\SpecialCharTok{+}
  \FunctionTok{scale\_fill\_manual}\NormalTok{(}\AttributeTok{values =}\NormalTok{ source\_colors) }\SpecialCharTok{+}
  \FunctionTok{labs}\NormalTok{(}
    \AttributeTok{title =} \StringTok{"Electricity Generation by Source – 12 June 2024"}\NormalTok{,}
    \AttributeTok{x =} \StringTok{"Hour"}\NormalTok{,}
    \AttributeTok{y =} \StringTok{"Generation (MW)"}\NormalTok{,}
    \AttributeTok{fill =} \StringTok{"Source"}
\NormalTok{  ) }\SpecialCharTok{+}
  \FunctionTok{theme\_minimal}\NormalTok{() }\SpecialCharTok{+}
  \FunctionTok{theme}\NormalTok{(}
    \AttributeTok{plot.title =} \FunctionTok{element\_text}\NormalTok{(}\AttributeTok{hjust =} \FloatTok{0.5}\NormalTok{, }\AttributeTok{face =} \StringTok{"bold"}\NormalTok{),}
    \AttributeTok{axis.text.x =} \FunctionTok{element\_text}\NormalTok{(}\AttributeTok{angle =} \DecValTok{45}\NormalTok{, }\AttributeTok{hjust =} \DecValTok{1}\NormalTok{)}
\NormalTok{  )}
\end{Highlighting}
\end{Shaded}

\pandocbounded{\includegraphics[keepaspectratio]{project_files/figure-pdf/unnamed-chunk-7-1.pdf}}

This chart clearly shows how electricity demand is met by different
generation sources at different hours of the day.

There are two major factors that influence electricity consumption, or
demand. The first is air temperature. In Türkiye, electricity demand
increases when the average daily temperature rises above or drops below
15°C, due to higher use of heating and cooling systems.

The second factor is more long-term: the country's level of industrial
activity and population size. As the population grows and industrial
production expands, electricity demand also rises.

The chart below displays the moving average electricity demand data for
2023 and 2024.

\begin{Shaded}
\begin{Highlighting}[]
\FunctionTok{library}\NormalTok{(dplyr)}
\FunctionTok{library}\NormalTok{(ggplot2)}
\FunctionTok{library}\NormalTok{(lubridate)}
\FunctionTok{library}\NormalTok{(zoo)}

\NormalTok{demand\_yoy }\OtherTok{\textless{}{-}}\NormalTok{ epias\_merged }\SpecialCharTok{\%\textgreater{}\%}
  \FunctionTok{filter}\NormalTok{(}\FunctionTok{year}\NormalTok{(date) }\SpecialCharTok{\%in\%} \FunctionTok{c}\NormalTok{(}\DecValTok{2023}\NormalTok{, }\DecValTok{2024}\NormalTok{)) }\SpecialCharTok{\%\textgreater{}\%}
  \FunctionTok{arrange}\NormalTok{(date, hour) }\SpecialCharTok{\%\textgreater{}\%}
  \FunctionTok{mutate}\NormalTok{(}
    \AttributeTok{year =} \FunctionTok{year}\NormalTok{(date),}
    \AttributeTok{doy =} \FunctionTok{yday}\NormalTok{(date),}
    \AttributeTok{datetime =} \FunctionTok{as.POSIXct}\NormalTok{(}\FunctionTok{paste}\NormalTok{(date, hour), }\AttributeTok{format =} \StringTok{"\%Y{-}\%m{-}\%d \%H:\%M"}\NormalTok{)}
\NormalTok{  ) }\SpecialCharTok{\%\textgreater{}\%}
  \FunctionTok{group\_by}\NormalTok{(year) }\SpecialCharTok{\%\textgreater{}\%}
  \FunctionTok{arrange}\NormalTok{(datetime) }\SpecialCharTok{\%\textgreater{}\%}
  \FunctionTok{mutate}\NormalTok{(}
    \AttributeTok{demand\_7day\_avg =} \FunctionTok{rollmean}\NormalTok{(demand, }\AttributeTok{k =} \DecValTok{24} \SpecialCharTok{*} \DecValTok{7}\NormalTok{, }\AttributeTok{fill =} \ConstantTok{NA}\NormalTok{, }\AttributeTok{align =} \StringTok{"right"}\NormalTok{)}
\NormalTok{  ) }\SpecialCharTok{\%\textgreater{}\%}
  \FunctionTok{ungroup}\NormalTok{()}

\FunctionTok{ggplot}\NormalTok{(demand\_yoy, }\FunctionTok{aes}\NormalTok{(}\AttributeTok{x =}\NormalTok{ doy, }\AttributeTok{y =}\NormalTok{ demand\_7day\_avg, }\AttributeTok{color =} \FunctionTok{as.factor}\NormalTok{(year))) }\SpecialCharTok{+}
  \FunctionTok{geom\_line}\NormalTok{(}\AttributeTok{size =} \FloatTok{1.2}\NormalTok{) }\SpecialCharTok{+}
  \FunctionTok{labs}\NormalTok{(}
    \AttributeTok{title =} \StringTok{"Year{-}over{-}Year Demand Comparison (7{-}Day Moving Average)"}\NormalTok{,}
    \AttributeTok{x =} \StringTok{"Day of Year"}\NormalTok{,}
    \AttributeTok{y =} \StringTok{"Demand (MW)"}\NormalTok{,}
    \AttributeTok{color =} \StringTok{"Year"}
\NormalTok{  ) }\SpecialCharTok{+}
  \FunctionTok{scale\_color\_manual}\NormalTok{(}\AttributeTok{values =} \FunctionTok{c}\NormalTok{(}\StringTok{"2023"} \OtherTok{=} \StringTok{"purple"}\NormalTok{, }\StringTok{"2024"} \OtherTok{=} \StringTok{"gold"}\NormalTok{)) }\SpecialCharTok{+}
  \FunctionTok{expand\_limits}\NormalTok{(}\AttributeTok{y =} \DecValTok{20000}\NormalTok{) }\SpecialCharTok{+}
  \FunctionTok{theme\_minimal}\NormalTok{() }\SpecialCharTok{+}
  \FunctionTok{theme}\NormalTok{(}
    \AttributeTok{plot.title =} \FunctionTok{element\_text}\NormalTok{(}\AttributeTok{hjust =} \FloatTok{0.5}\NormalTok{, }\AttributeTok{face =} \StringTok{"bold"}\NormalTok{)}
\NormalTok{  )}
\end{Highlighting}
\end{Shaded}

\pandocbounded{\includegraphics[keepaspectratio]{project_files/figure-pdf/unnamed-chunk-8-1.pdf}}

As seen in the chart, electricity demand increases during the summer due
to rising temperatures, and also in winter as temperatures drop.
Additionally, demand tends to reach its lowest levels during religious
holidays, when industrial activity comes to a near halt.

When examining demand data at a higher resolution, it becomes clear how
electricity demand varies across the days of the week.

\begin{Shaded}
\begin{Highlighting}[]
\FunctionTok{library}\NormalTok{(dplyr)}
\FunctionTok{library}\NormalTok{(ggplot2)}
\FunctionTok{library}\NormalTok{(lubridate)}

\NormalTok{nov\_demand }\OtherTok{\textless{}{-}}\NormalTok{ epias\_merged }\SpecialCharTok{\%\textgreater{}\%}
  \FunctionTok{filter}\NormalTok{(date }\SpecialCharTok{\textgreater{}=} \FunctionTok{as.Date}\NormalTok{(}\StringTok{"2024{-}11{-}01"}\NormalTok{) }\SpecialCharTok{\&}\NormalTok{ date }\SpecialCharTok{\textless{}=} \FunctionTok{as.Date}\NormalTok{(}\StringTok{"2024{-}11{-}10"}\NormalTok{)) }\SpecialCharTok{\%\textgreater{}\%}
  \FunctionTok{mutate}\NormalTok{(}
    \AttributeTok{datetime =} \FunctionTok{as.POSIXct}\NormalTok{(}\FunctionTok{paste}\NormalTok{(date, hour), }\AttributeTok{format =} \StringTok{"\%Y{-}\%m{-}\%d \%H:\%M"}\NormalTok{),}
    \AttributeTok{day\_num =} \FunctionTok{day}\NormalTok{(date) }\SpecialCharTok{+} \FunctionTok{hour}\NormalTok{(datetime) }\SpecialCharTok{/} \DecValTok{24} 
\NormalTok{  )}

\FunctionTok{ggplot}\NormalTok{(nov\_demand, }\FunctionTok{aes}\NormalTok{(}\AttributeTok{x =}\NormalTok{ day\_num, }\AttributeTok{y =}\NormalTok{ demand)) }\SpecialCharTok{+}
  \FunctionTok{geom\_line}\NormalTok{(}\AttributeTok{color =} \StringTok{"steelblue"}\NormalTok{, }\AttributeTok{size =} \FloatTok{1.2}\NormalTok{) }\SpecialCharTok{+}
  \FunctionTok{scale\_x\_continuous}\NormalTok{(}\AttributeTok{breaks =} \DecValTok{1}\SpecialCharTok{:}\DecValTok{10}\NormalTok{) }\SpecialCharTok{+}
  \FunctionTok{labs}\NormalTok{(}
    \AttributeTok{title =} \StringTok{"November 2024 {-} First 10 Days Hourly Demand"}\NormalTok{,}
    \AttributeTok{x =} \StringTok{"Day of Month"}\NormalTok{,}
    \AttributeTok{y =} \StringTok{"Demand (MW)"}
\NormalTok{  ) }\SpecialCharTok{+}
  \FunctionTok{theme\_minimal}\NormalTok{() }\SpecialCharTok{+}
  \FunctionTok{theme}\NormalTok{(}
    \AttributeTok{plot.title =} \FunctionTok{element\_text}\NormalTok{(}\AttributeTok{hjust =} \FloatTok{0.5}\NormalTok{, }\AttributeTok{face =} \StringTok{"bold"}\NormalTok{)}
\NormalTok{  )}
\end{Highlighting}
\end{Shaded}

\pandocbounded{\includegraphics[keepaspectratio]{project_files/figure-pdf/unnamed-chunk-9-1.pdf}}

Electricity demand remains relatively consistent during weekdays,
whereas a noticeable drop is observed on Saturdays and Sundays.

A comprehensive understanding of the data requires analyzing generation
figures in conjunction with demand.

Renewable energy sources exhibit different generation trend
characteristics. Among these, the most influential factors are
meteorological conditions and seasonal effects. Below is the generation
data from various sources for a single day. As seen, solar production
peaks around midday and drops to zero after sunset. Wind generation, on
the other hand, may display varying patterns from day to day.

\begin{Shaded}
\begin{Highlighting}[]
\FunctionTok{library}\NormalTok{(dplyr)}
\FunctionTok{library}\NormalTok{(ggplot2)}
\FunctionTok{library}\NormalTok{(lubridate)}

\NormalTok{nov\_renew }\OtherTok{\textless{}{-}}\NormalTok{ epias\_merged }\SpecialCharTok{\%\textgreater{}\%}
  \FunctionTok{filter}\NormalTok{(date }\SpecialCharTok{\textgreater{}=} \FunctionTok{as.Date}\NormalTok{(}\StringTok{"2024{-}11{-}01"}\NormalTok{) }\SpecialCharTok{\&}\NormalTok{ date }\SpecialCharTok{\textless{}=} \FunctionTok{as.Date}\NormalTok{(}\StringTok{"2024{-}11{-}10"}\NormalTok{)) }\SpecialCharTok{\%\textgreater{}\%}
  \FunctionTok{mutate}\NormalTok{(}
    \AttributeTok{datetime =} \FunctionTok{as.POSIXct}\NormalTok{(}\FunctionTok{paste}\NormalTok{(date, hour), }\AttributeTok{format =} \StringTok{"\%Y{-}\%m{-}\%d \%H:\%M"}\NormalTok{)}
\NormalTok{  )}

\FunctionTok{ggplot}\NormalTok{(nov\_renew, }\FunctionTok{aes}\NormalTok{(}\AttributeTok{x =}\NormalTok{ datetime)) }\SpecialCharTok{+}
  \FunctionTok{geom\_line}\NormalTok{(}\FunctionTok{aes}\NormalTok{(}\AttributeTok{y =}\NormalTok{ solar, }\AttributeTok{color =} \StringTok{"Solar"}\NormalTok{), }\AttributeTok{size =} \FloatTok{1.2}\NormalTok{) }\SpecialCharTok{+}
  \FunctionTok{geom\_line}\NormalTok{(}\FunctionTok{aes}\NormalTok{(}\AttributeTok{y =}\NormalTok{ wind, }\AttributeTok{color =} \StringTok{"Wind"}\NormalTok{), }\AttributeTok{size =} \FloatTok{1.2}\NormalTok{) }\SpecialCharTok{+}
  \FunctionTok{scale\_color\_manual}\NormalTok{(}\AttributeTok{values =} \FunctionTok{c}\NormalTok{(}\StringTok{"Solar"} \OtherTok{=} \StringTok{"orange"}\NormalTok{, }\StringTok{"Wind"} \OtherTok{=} \StringTok{"darkgreen"}\NormalTok{)) }\SpecialCharTok{+}
  \FunctionTok{labs}\NormalTok{(}
    \AttributeTok{title =} \StringTok{"Solar and Wind Generation – November 1–10, 2024"}\NormalTok{,}
    \AttributeTok{x =} \StringTok{"Datetime"}\NormalTok{,}
    \AttributeTok{y =} \StringTok{"Generation (MW)"}\NormalTok{,}
    \AttributeTok{color =} \StringTok{"Source"}
\NormalTok{  ) }\SpecialCharTok{+}
  \FunctionTok{theme\_minimal}\NormalTok{() }\SpecialCharTok{+}
  \FunctionTok{theme}\NormalTok{(}
    \AttributeTok{plot.title =} \FunctionTok{element\_text}\NormalTok{(}\AttributeTok{hjust =} \FloatTok{0.5}\NormalTok{, }\AttributeTok{face =} \StringTok{"bold"}\NormalTok{)}
\NormalTok{  )}
\end{Highlighting}
\end{Shaded}

\pandocbounded{\includegraphics[keepaspectratio]{project_files/figure-pdf/unnamed-chunk-10-1.pdf}}

An analysis of solar and run-of-river generation data reveals clear
seasonal trends. While solar production increases during the summer
months, run-of-river generation peaks in the spring.

\begin{Shaded}
\begin{Highlighting}[]
\FunctionTok{library}\NormalTok{(dplyr)}
\FunctionTok{library}\NormalTok{(ggplot2)}
\FunctionTok{library}\NormalTok{(zoo)}
\FunctionTok{library}\NormalTok{(lubridate)}

\NormalTok{solar\_river\_daily }\OtherTok{\textless{}{-}}\NormalTok{ epias\_merged }\SpecialCharTok{\%\textgreater{}\%}
  \FunctionTok{mutate}\NormalTok{(}\AttributeTok{date =} \FunctionTok{as.Date}\NormalTok{(date)) }\SpecialCharTok{\%\textgreater{}\%}
  \FunctionTok{group\_by}\NormalTok{(date) }\SpecialCharTok{\%\textgreater{}\%}
  \FunctionTok{summarise}\NormalTok{(}
    \AttributeTok{solar =} \FunctionTok{mean}\NormalTok{(solar, }\AttributeTok{na.rm =} \ConstantTok{TRUE}\NormalTok{),}
    \AttributeTok{runofriver =} \FunctionTok{mean}\NormalTok{(runofriver, }\AttributeTok{na.rm =} \ConstantTok{TRUE}\NormalTok{)}
\NormalTok{  ) }\SpecialCharTok{\%\textgreater{}\%}
  \FunctionTok{filter}\NormalTok{(}\FunctionTok{year}\NormalTok{(date) }\SpecialCharTok{==} \DecValTok{2023}\NormalTok{) }\SpecialCharTok{\%\textgreater{}\%}
  \FunctionTok{arrange}\NormalTok{(date) }\SpecialCharTok{\%\textgreater{}\%}
  \FunctionTok{mutate}\NormalTok{(}
    \AttributeTok{solar\_ma =} \FunctionTok{rollmean}\NormalTok{(solar, }\AttributeTok{k =} \DecValTok{7}\NormalTok{, }\AttributeTok{fill =} \ConstantTok{NA}\NormalTok{, }\AttributeTok{align =} \StringTok{"right"}\NormalTok{),}
    \AttributeTok{runofriver\_ma =} \FunctionTok{rollmean}\NormalTok{(runofriver, }\AttributeTok{k =} \DecValTok{7}\NormalTok{, }\AttributeTok{fill =} \ConstantTok{NA}\NormalTok{, }\AttributeTok{align =} \StringTok{"right"}\NormalTok{)}
\NormalTok{  )}

\FunctionTok{ggplot}\NormalTok{(solar\_river\_daily, }\FunctionTok{aes}\NormalTok{(}\AttributeTok{x =}\NormalTok{ date)) }\SpecialCharTok{+}
  \FunctionTok{geom\_line}\NormalTok{(}\FunctionTok{aes}\NormalTok{(}\AttributeTok{y =}\NormalTok{ solar\_ma, }\AttributeTok{color =} \StringTok{"Solar"}\NormalTok{), }\AttributeTok{size =} \FloatTok{1.2}\NormalTok{) }\SpecialCharTok{+}
  \FunctionTok{geom\_line}\NormalTok{(}\FunctionTok{aes}\NormalTok{(}\AttributeTok{y =}\NormalTok{ runofriver\_ma, }\AttributeTok{color =} \StringTok{"Run{-}of{-}River"}\NormalTok{), }\AttributeTok{size =} \FloatTok{1.2}\NormalTok{) }\SpecialCharTok{+}
  \FunctionTok{scale\_color\_manual}\NormalTok{(}\AttributeTok{values =} \FunctionTok{c}\NormalTok{(}\StringTok{"Solar"} \OtherTok{=} \StringTok{"orange"}\NormalTok{, }\StringTok{"Run{-}of{-}River"} \OtherTok{=} \StringTok{"steelblue"}\NormalTok{)) }\SpecialCharTok{+}
  \FunctionTok{labs}\NormalTok{(}
    \AttributeTok{title =} \StringTok{"7{-}Day Moving Average of Solar and Run{-}of{-}River Generation – 2023"}\NormalTok{,}
    \AttributeTok{x =} \StringTok{"Date"}\NormalTok{,}
    \AttributeTok{y =} \StringTok{"Generation (MWh)"}\NormalTok{,}
    \AttributeTok{color =} \StringTok{"Source"}
\NormalTok{  ) }\SpecialCharTok{+}
  \FunctionTok{theme\_minimal}\NormalTok{() }\SpecialCharTok{+}
  \FunctionTok{theme}\NormalTok{(}\AttributeTok{plot.title =} \FunctionTok{element\_text}\NormalTok{(}\AttributeTok{hjust =} \FloatTok{0.5}\NormalTok{, }\AttributeTok{face =} \StringTok{"bold"}\NormalTok{))}
\end{Highlighting}
\end{Shaded}

\pandocbounded{\includegraphics[keepaspectratio]{project_files/figure-pdf/unnamed-chunk-11-1.pdf}}

\subsection{3.3 Model Fitting}\label{model-fitting}

Daily average electricity prices were forecasted using a multiple linear
regression approach.

Two different models were developed. Although both models followed the
same computational logic, they differed in feature selection and feature
aggregation strategies. In the first model, all available variables were
used individually as features, whereas in the second model, a feature
aggregation approach was applied.

3.3.1 What Happens If We Don't Aggregate Features?

\begin{Shaded}
\begin{Highlighting}[]
\FunctionTok{library}\NormalTok{(dplyr)}
\FunctionTok{library}\NormalTok{(car)}
\end{Highlighting}
\end{Shaded}

\begin{verbatim}
Zorunlu paket yükleniyor: carData
\end{verbatim}

\begin{verbatim}

Attaching package: 'car'
\end{verbatim}

\begin{verbatim}
The following object is masked from 'package:dplyr':

    recode
\end{verbatim}

\begin{Shaded}
\begin{Highlighting}[]
\FunctionTok{library}\NormalTok{(ggplot2)}

\NormalTok{reg\_data1 }\OtherTok{\textless{}{-}}\NormalTok{ epias\_daily }\SpecialCharTok{\%\textgreater{}\%}
  \FunctionTok{select}\NormalTok{(price, naturalgas, wind, lignite, darkcoal, importedcoal, fueloil, geothermal, dam, naphta, biomass, runofriver, other, demand, solar, natgasprice) }\SpecialCharTok{\%\textgreater{}\%}
  \FunctionTok{na.omit}\NormalTok{()}

\NormalTok{model1 }\OtherTok{\textless{}{-}} \FunctionTok{lm}\NormalTok{(price }\SpecialCharTok{\textasciitilde{}}\NormalTok{ naturalgas }\SpecialCharTok{+}\NormalTok{ wind }\SpecialCharTok{+}\NormalTok{ lignite }\SpecialCharTok{+}\NormalTok{ darkcoal }\SpecialCharTok{+}\NormalTok{ importedcoal }\SpecialCharTok{+}\NormalTok{ fueloil }\SpecialCharTok{+}\NormalTok{ geothermal }\SpecialCharTok{+}\NormalTok{ dam }\SpecialCharTok{+}\NormalTok{ naphta }\SpecialCharTok{+}\NormalTok{ biomass }\SpecialCharTok{+}\NormalTok{ runofriver }\SpecialCharTok{+}\NormalTok{ other }\SpecialCharTok{+}\NormalTok{ demand }\SpecialCharTok{+}\NormalTok{ solar }\SpecialCharTok{+}\NormalTok{ natgasprice, }\AttributeTok{data =}\NormalTok{ reg\_data1)}

\FunctionTok{summary}\NormalTok{(model1)}
\end{Highlighting}
\end{Shaded}

\begin{verbatim}

Call:
lm(formula = price ~ naturalgas + wind + lignite + darkcoal + 
    importedcoal + fueloil + geothermal + dam + naphta + biomass + 
    runofriver + other + demand + solar + natgasprice, data = reg_data1)

Residuals:
    Min      1Q  Median      3Q     Max 
-913.47  -89.95   -1.53  101.26  501.77 

Coefficients:
               Estimate Std. Error t value Pr(>|t|)    
(Intercept)  -1.982e+03  2.039e+02  -9.725  < 2e-16 ***
naturalgas    4.853e-02  1.083e-02   4.481 8.63e-06 ***
wind         -4.299e-02  1.021e-02  -4.212 2.85e-05 ***
lignite       1.250e-02  2.292e-02   0.545  0.58566    
darkcoal      1.293e-01  9.810e-02   1.318  0.18779    
importedcoal  7.100e-02  1.422e-02   4.993 7.47e-07 ***
fueloil       2.881e+00  1.219e+00   2.364  0.01834 *  
geothermal    2.114e-01  1.384e-01   1.528  0.12707    
dam          -5.017e-02  1.136e-02  -4.415 1.16e-05 ***
naphta       -4.473e+01  6.276e+01  -0.713  0.47623    
biomass       8.360e-04  1.765e-01   0.005  0.99622    
runofriver    1.168e-01  1.824e-02   6.404 2.74e-10 ***
other        -4.118e-02  8.859e-02  -0.465  0.64217    
demand        3.037e-02  9.991e-03   3.040  0.00245 ** 
solar         2.519e-02  1.683e-02   1.496  0.13501    
natgasprice   1.701e-01  6.133e-03  27.736  < 2e-16 ***
---
Signif. codes:  0 '***' 0.001 '**' 0.01 '*' 0.05 '.' 0.1 ' ' 1

Residual standard error: 168.9 on 715 degrees of freedom
Multiple R-squared:  0.8836,    Adjusted R-squared:  0.8812 
F-statistic: 361.9 on 15 and 715 DF,  p-value: < 2.2e-16
\end{verbatim}

\begin{Shaded}
\begin{Highlighting}[]
\FunctionTok{vif}\NormalTok{(model1)}
\end{Highlighting}
\end{Shaded}

\begin{verbatim}
  naturalgas         wind      lignite     darkcoal importedcoal      fueloil 
   36.019212    11.420398     3.354866     1.503104    13.072816     1.279864 
  geothermal          dam       naphta      biomass   runofriver        other 
    5.939204    10.702876     1.057892     2.303100    11.131208     4.756550 
      demand        solar  natgasprice 
   51.610155     8.003321     2.753763 
\end{verbatim}

\begin{Shaded}
\begin{Highlighting}[]
\FunctionTok{par}\NormalTok{(}\AttributeTok{mfrow =} \FunctionTok{c}\NormalTok{(}\DecValTok{2}\NormalTok{, }\DecValTok{2}\NormalTok{))}
\FunctionTok{plot}\NormalTok{(model1)}
\end{Highlighting}
\end{Shaded}

\pandocbounded{\includegraphics[keepaspectratio]{project_files/figure-pdf/unnamed-chunk-12-1.pdf}}

\begin{Shaded}
\begin{Highlighting}[]
\FunctionTok{par}\NormalTok{(}\AttributeTok{mfrow =} \FunctionTok{c}\NormalTok{(}\DecValTok{1}\NormalTok{, }\DecValTok{1}\NormalTok{))}


\NormalTok{reg\_data1}\SpecialCharTok{$}\NormalTok{predicted\_price1 }\OtherTok{\textless{}{-}} \FunctionTok{predict}\NormalTok{(model1)}

\FunctionTok{ggplot}\NormalTok{(reg\_data1, }\FunctionTok{aes}\NormalTok{(}\AttributeTok{x =}\NormalTok{ predicted\_price1, }\AttributeTok{y =}\NormalTok{ price)) }\SpecialCharTok{+}
  \FunctionTok{geom\_point}\NormalTok{(}\AttributeTok{color =} \StringTok{"steelblue"}\NormalTok{) }\SpecialCharTok{+}
  \FunctionTok{geom\_abline}\NormalTok{(}\AttributeTok{slope =} \DecValTok{1}\NormalTok{, }\AttributeTok{intercept =} \DecValTok{0}\NormalTok{, }\AttributeTok{color =} \StringTok{"darkred"}\NormalTok{, }\AttributeTok{linetype =} \StringTok{"dashed"}\NormalTok{) }\SpecialCharTok{+}
  \FunctionTok{labs}\NormalTok{(}
    \AttributeTok{title =} \StringTok{"Actual vs Predicted Electricity Price"}\NormalTok{,}
    \AttributeTok{x =} \StringTok{"Predicted Price"}\NormalTok{,}
    \AttributeTok{y =} \StringTok{"Actual Price"}
\NormalTok{  ) }\SpecialCharTok{+}
  \FunctionTok{theme\_minimal}\NormalTok{()}
\end{Highlighting}
\end{Shaded}

\pandocbounded{\includegraphics[keepaspectratio]{project_files/figure-pdf/unnamed-chunk-12-2.pdf}}

3.3.2 What If We Aggregate Features?

\begin{Shaded}
\begin{Highlighting}[]
\FunctionTok{library}\NormalTok{(dplyr)}
\FunctionTok{library}\NormalTok{(car)}
\FunctionTok{library}\NormalTok{(ggplot2)}

\NormalTok{reg\_data }\OtherTok{\textless{}{-}}\NormalTok{ epias\_simplified\_daily }\SpecialCharTok{\%\textgreater{}\%}
  \FunctionTok{select}\NormalTok{(price, cheap\_thermal, naturalgas, importedcoal, renewables, demand, solar, dam, natgasprice) }\SpecialCharTok{\%\textgreater{}\%}
  \FunctionTok{na.omit}\NormalTok{()}


\NormalTok{model }\OtherTok{\textless{}{-}} \FunctionTok{lm}\NormalTok{(price }\SpecialCharTok{\textasciitilde{}}\NormalTok{ cheap\_thermal }\SpecialCharTok{+}\NormalTok{ naturalgas }\SpecialCharTok{+}\NormalTok{ importedcoal }\SpecialCharTok{+}\NormalTok{ renewables }\SpecialCharTok{+}\NormalTok{ demand }\SpecialCharTok{+}\NormalTok{ solar }\SpecialCharTok{+}\NormalTok{ dam }\SpecialCharTok{+}\NormalTok{ natgasprice, }\AttributeTok{data =}\NormalTok{ reg\_data)}

\FunctionTok{summary}\NormalTok{(model)}
\end{Highlighting}
\end{Shaded}

\begin{verbatim}

Call:
lm(formula = price ~ cheap_thermal + naturalgas + importedcoal + 
    renewables + demand + solar + dam + natgasprice, data = reg_data)

Residuals:
     Min       1Q   Median       3Q      Max 
-1078.97   -97.87     7.74   114.92   587.23 

Coefficients:
                Estimate Std. Error t value Pr(>|t|)    
(Intercept)   -6.994e+02  1.027e+02  -6.812 2.03e-11 ***
cheap_thermal  5.054e-03  2.295e-02   0.220 0.825754    
naturalgas     1.977e-02  1.026e-02   1.926 0.054483 .  
importedcoal  -3.961e-02  1.080e-02  -3.666 0.000264 ***
renewables    -7.091e-02  9.712e-03  -7.302 7.52e-13 ***
demand         5.646e-02  8.919e-03   6.330 4.29e-10 ***
solar         -4.154e-02  1.038e-02  -4.003 6.91e-05 ***
dam           -4.511e-02  1.023e-02  -4.412 1.18e-05 ***
natgasprice    1.544e-01  4.778e-03  32.311  < 2e-16 ***
---
Signif. codes:  0 '***' 0.001 '**' 0.01 '*' 0.05 '.' 0.1 ' ' 1

Residual standard error: 191.1 on 722 degrees of freedom
Multiple R-squared:  0.8494,    Adjusted R-squared:  0.8477 
F-statistic: 509.1 on 8 and 722 DF,  p-value: < 2.2e-16
\end{verbatim}

\begin{Shaded}
\begin{Highlighting}[]
\FunctionTok{vif}\NormalTok{(model)}
\end{Highlighting}
\end{Shaded}

\begin{verbatim}
cheap_thermal    naturalgas  importedcoal    renewables        demand 
     2.984234     25.250485      5.887888      9.567100     32.098596 
        solar           dam   natgasprice 
     2.374800      6.767137      1.304383 
\end{verbatim}

\begin{Shaded}
\begin{Highlighting}[]
\FunctionTok{par}\NormalTok{(}\AttributeTok{mfrow =} \FunctionTok{c}\NormalTok{(}\DecValTok{2}\NormalTok{, }\DecValTok{2}\NormalTok{))}
\FunctionTok{plot}\NormalTok{(model)}
\end{Highlighting}
\end{Shaded}

\pandocbounded{\includegraphics[keepaspectratio]{project_files/figure-pdf/unnamed-chunk-13-1.pdf}}

\begin{Shaded}
\begin{Highlighting}[]
\FunctionTok{par}\NormalTok{(}\AttributeTok{mfrow =} \FunctionTok{c}\NormalTok{(}\DecValTok{1}\NormalTok{, }\DecValTok{1}\NormalTok{))}


\NormalTok{reg\_data}\SpecialCharTok{$}\NormalTok{predicted\_price }\OtherTok{\textless{}{-}} \FunctionTok{predict}\NormalTok{(model)}

\FunctionTok{ggplot}\NormalTok{(reg\_data, }\FunctionTok{aes}\NormalTok{(}\AttributeTok{x =}\NormalTok{ predicted\_price, }\AttributeTok{y =}\NormalTok{ price)) }\SpecialCharTok{+}
  \FunctionTok{geom\_point}\NormalTok{(}\AttributeTok{color =} \StringTok{"steelblue"}\NormalTok{) }\SpecialCharTok{+}
  \FunctionTok{geom\_abline}\NormalTok{(}\AttributeTok{slope =} \DecValTok{1}\NormalTok{, }\AttributeTok{intercept =} \DecValTok{0}\NormalTok{, }\AttributeTok{color =} \StringTok{"darkred"}\NormalTok{, }\AttributeTok{linetype =} \StringTok{"dashed"}\NormalTok{) }\SpecialCharTok{+}
  \FunctionTok{labs}\NormalTok{(}
    \AttributeTok{title =} \StringTok{"Actual vs Predicted Electricity Price"}\NormalTok{,}
    \AttributeTok{x =} \StringTok{"Predicted Price"}\NormalTok{,}
    \AttributeTok{y =} \StringTok{"Actual Price"}
\NormalTok{  ) }\SpecialCharTok{+}
  \FunctionTok{theme\_minimal}\NormalTok{()}
\end{Highlighting}
\end{Shaded}

\pandocbounded{\includegraphics[keepaspectratio]{project_files/figure-pdf/unnamed-chunk-13-2.pdf}}

\subsection{3.4 Results}\label{results}

When both models are evaluated, the model without feature aggregation
shows a higher R-squared value and a lower standard error, indicating
better predictive performance. However, an inspection of the feature
coefficients reveals that the first model includes more features with
statistically insignificant coefficients. Furthermore, the near-zero
effect of solar generation in this model raises questions about its
reliability.

Although the second model exhibits a slightly higher error rate, it
offers a clearer explanation of how each feature impacts the electricity
price and stands out for its simpler structure. Despite this, both
models appear to produce somewhat biased predictions for extreme price
values (very high or very low), likely due to the influence of outliers.

The chart below presents the actual and predicted electricity prices for
November and June 2024. While both models demonstrate a strong ability
to track the general trend, Model 2 shows a greater tendency toward
producing outlier or extreme values.

\begin{Shaded}
\begin{Highlighting}[]
\NormalTok{data1 }\OtherTok{\textless{}{-}}\NormalTok{ epias\_daily }\SpecialCharTok{\%\textgreater{}\%}
  \FunctionTok{select}\NormalTok{(date, price, naturalgas, wind, lignite, darkcoal, importedcoal, fueloil, geothermal, dam, naphta, biomass, runofriver, other, demand, solar, natgasprice) }\SpecialCharTok{\%\textgreater{}\%}
  \FunctionTok{na.omit}\NormalTok{() }\SpecialCharTok{\%\textgreater{}\%}
  \FunctionTok{mutate}\NormalTok{(}\AttributeTok{predicted\_price1 =} \FunctionTok{predict}\NormalTok{(model1, }\AttributeTok{newdata =}\NormalTok{ .))}

\NormalTok{data2 }\OtherTok{\textless{}{-}}\NormalTok{ epias\_simplified\_daily }\SpecialCharTok{\%\textgreater{}\%}
  \FunctionTok{select}\NormalTok{(date, price, cheap\_thermal, naturalgas, importedcoal, renewables, demand, solar, dam, natgasprice) }\SpecialCharTok{\%\textgreater{}\%}
  \FunctionTok{na.omit}\NormalTok{() }\SpecialCharTok{\%\textgreater{}\%}
  \FunctionTok{mutate}\NormalTok{(}\AttributeTok{predicted\_price2 =} \FunctionTok{predict}\NormalTok{(model, }\AttributeTok{newdata =}\NormalTok{ .))}


\NormalTok{combined }\OtherTok{\textless{}{-}}\NormalTok{ data1 }\SpecialCharTok{\%\textgreater{}\%}
  \FunctionTok{inner\_join}\NormalTok{(data2 }\SpecialCharTok{\%\textgreater{}\%} \FunctionTok{select}\NormalTok{(date, predicted\_price2), }\AttributeTok{by =} \StringTok{"date"}\NormalTok{) }\SpecialCharTok{\%\textgreater{}\%}
  \FunctionTok{filter}\NormalTok{(date }\SpecialCharTok{\textgreater{}=} \FunctionTok{as.Date}\NormalTok{(}\StringTok{"2024{-}11{-}01"}\NormalTok{) }\SpecialCharTok{\&}\NormalTok{ date }\SpecialCharTok{\textless{}=} \FunctionTok{as.Date}\NormalTok{(}\StringTok{"2024{-}11{-}30"}\NormalTok{))}


\FunctionTok{library}\NormalTok{(tidyr)}
\NormalTok{plot\_data }\OtherTok{\textless{}{-}}\NormalTok{ combined }\SpecialCharTok{\%\textgreater{}\%}
  \FunctionTok{select}\NormalTok{(date, price, predicted\_price1, predicted\_price2) }\SpecialCharTok{\%\textgreater{}\%}
  \FunctionTok{pivot\_longer}\NormalTok{(}\AttributeTok{cols =} \FunctionTok{c}\NormalTok{(price, predicted\_price1, predicted\_price2),}
               \AttributeTok{names\_to =} \StringTok{"type"}\NormalTok{, }\AttributeTok{values\_to =} \StringTok{"value"}\NormalTok{)}


\FunctionTok{ggplot}\NormalTok{(plot\_data, }\FunctionTok{aes}\NormalTok{(}\AttributeTok{x =}\NormalTok{ date, }\AttributeTok{y =}\NormalTok{ value, }\AttributeTok{color =}\NormalTok{ type)) }\SpecialCharTok{+}
  \FunctionTok{geom\_line}\NormalTok{(}\AttributeTok{size =} \FloatTok{1.2}\NormalTok{) }\SpecialCharTok{+}
  \FunctionTok{scale\_color\_manual}\NormalTok{(}
    \AttributeTok{values =} \FunctionTok{c}\NormalTok{(}\StringTok{"price"} \OtherTok{=} \StringTok{"black"}\NormalTok{, }\StringTok{"predicted\_price1"} \OtherTok{=} \StringTok{"steelblue"}\NormalTok{, }\StringTok{"predicted\_price2"} \OtherTok{=} \StringTok{"orange"}\NormalTok{),}
    \AttributeTok{labels =} \FunctionTok{c}\NormalTok{(}\StringTok{"Actual Price"}\NormalTok{, }\StringTok{"Model 1 Prediction"}\NormalTok{, }\StringTok{"Model 2 Prediction"}\NormalTok{)}
\NormalTok{  ) }\SpecialCharTok{+}
  \FunctionTok{labs}\NormalTok{(}
    \AttributeTok{title =} \StringTok{"Electricity Price Forecast Actual vs Model 1 \& 2"}\NormalTok{,}
    \AttributeTok{x =} \StringTok{"Date"}\NormalTok{,}
    \AttributeTok{y =} \StringTok{"Price (TL/MWh)"}\NormalTok{,}
    \AttributeTok{color =} \StringTok{"Legend"}
\NormalTok{  ) }\SpecialCharTok{+}
  \FunctionTok{theme\_minimal}\NormalTok{() }\SpecialCharTok{+}
  \FunctionTok{theme}\NormalTok{(}\AttributeTok{plot.title =} \FunctionTok{element\_text}\NormalTok{(}\AttributeTok{hjust =} \FloatTok{0.5}\NormalTok{, }\AttributeTok{face =} \StringTok{"bold"}\NormalTok{))}
\end{Highlighting}
\end{Shaded}

\pandocbounded{\includegraphics[keepaspectratio]{project_files/figure-pdf/unnamed-chunk-14-1.pdf}}

\begin{Shaded}
\begin{Highlighting}[]
\NormalTok{data1 }\OtherTok{\textless{}{-}}\NormalTok{ epias\_daily }\SpecialCharTok{\%\textgreater{}\%}
  \FunctionTok{select}\NormalTok{(date, price, naturalgas, wind, lignite, darkcoal, importedcoal, fueloil, geothermal, dam, naphta, biomass, runofriver, other, demand, solar, natgasprice) }\SpecialCharTok{\%\textgreater{}\%}
  \FunctionTok{na.omit}\NormalTok{() }\SpecialCharTok{\%\textgreater{}\%}
  \FunctionTok{mutate}\NormalTok{(}\AttributeTok{predicted\_price1 =} \FunctionTok{predict}\NormalTok{(model1, }\AttributeTok{newdata =}\NormalTok{ .))}

\NormalTok{data2 }\OtherTok{\textless{}{-}}\NormalTok{ epias\_simplified\_daily }\SpecialCharTok{\%\textgreater{}\%}
  \FunctionTok{select}\NormalTok{(date, price, cheap\_thermal, naturalgas, importedcoal, renewables, demand, solar, dam, natgasprice) }\SpecialCharTok{\%\textgreater{}\%}
  \FunctionTok{na.omit}\NormalTok{() }\SpecialCharTok{\%\textgreater{}\%}
  \FunctionTok{mutate}\NormalTok{(}\AttributeTok{predicted\_price2 =} \FunctionTok{predict}\NormalTok{(model, }\AttributeTok{newdata =}\NormalTok{ .))}


\NormalTok{combined }\OtherTok{\textless{}{-}}\NormalTok{ data1 }\SpecialCharTok{\%\textgreater{}\%}
  \FunctionTok{inner\_join}\NormalTok{(data2 }\SpecialCharTok{\%\textgreater{}\%} \FunctionTok{select}\NormalTok{(date, predicted\_price2), }\AttributeTok{by =} \StringTok{"date"}\NormalTok{) }\SpecialCharTok{\%\textgreater{}\%}
  \FunctionTok{filter}\NormalTok{(date }\SpecialCharTok{\textgreater{}=} \FunctionTok{as.Date}\NormalTok{(}\StringTok{"2024{-}06{-}01"}\NormalTok{) }\SpecialCharTok{\&}\NormalTok{ date }\SpecialCharTok{\textless{}=} \FunctionTok{as.Date}\NormalTok{(}\StringTok{"2024{-}06{-}30"}\NormalTok{))}


\FunctionTok{library}\NormalTok{(tidyr)}
\NormalTok{plot\_data }\OtherTok{\textless{}{-}}\NormalTok{ combined }\SpecialCharTok{\%\textgreater{}\%}
  \FunctionTok{select}\NormalTok{(date, price, predicted\_price1, predicted\_price2) }\SpecialCharTok{\%\textgreater{}\%}
  \FunctionTok{pivot\_longer}\NormalTok{(}\AttributeTok{cols =} \FunctionTok{c}\NormalTok{(price, predicted\_price1, predicted\_price2),}
               \AttributeTok{names\_to =} \StringTok{"type"}\NormalTok{, }\AttributeTok{values\_to =} \StringTok{"value"}\NormalTok{)}


\FunctionTok{ggplot}\NormalTok{(plot\_data, }\FunctionTok{aes}\NormalTok{(}\AttributeTok{x =}\NormalTok{ date, }\AttributeTok{y =}\NormalTok{ value, }\AttributeTok{color =}\NormalTok{ type)) }\SpecialCharTok{+}
  \FunctionTok{geom\_line}\NormalTok{(}\AttributeTok{size =} \FloatTok{1.2}\NormalTok{) }\SpecialCharTok{+}
  \FunctionTok{scale\_color\_manual}\NormalTok{(}
    \AttributeTok{values =} \FunctionTok{c}\NormalTok{(}\StringTok{"price"} \OtherTok{=} \StringTok{"black"}\NormalTok{, }\StringTok{"predicted\_price1"} \OtherTok{=} \StringTok{"steelblue"}\NormalTok{, }\StringTok{"predicted\_price2"} \OtherTok{=} \StringTok{"orange"}\NormalTok{),}
    \AttributeTok{labels =} \FunctionTok{c}\NormalTok{(}\StringTok{"Actual Price"}\NormalTok{, }\StringTok{"Model 1 Prediction"}\NormalTok{, }\StringTok{"Model 2 Prediction"}\NormalTok{)}
\NormalTok{  ) }\SpecialCharTok{+}
  \FunctionTok{labs}\NormalTok{(}
    \AttributeTok{title =} \StringTok{"Electricity Price Forecast Actual vs Model 1 \& 2"}\NormalTok{,}
    \AttributeTok{x =} \StringTok{"Date"}\NormalTok{,}
    \AttributeTok{y =} \StringTok{"Price (TL/MWh)"}\NormalTok{,}
    \AttributeTok{color =} \StringTok{"Legend"}
\NormalTok{  ) }\SpecialCharTok{+}
  \FunctionTok{theme\_minimal}\NormalTok{() }\SpecialCharTok{+}
  \FunctionTok{theme}\NormalTok{(}\AttributeTok{plot.title =} \FunctionTok{element\_text}\NormalTok{(}\AttributeTok{hjust =} \FloatTok{0.5}\NormalTok{, }\AttributeTok{face =} \StringTok{"bold"}\NormalTok{))}
\end{Highlighting}
\end{Shaded}

\pandocbounded{\includegraphics[keepaspectratio]{project_files/figure-pdf/unnamed-chunk-15-1.pdf}}

\begin{Shaded}
\begin{Highlighting}[]
\FunctionTok{library}\NormalTok{(dplyr)}
\FunctionTok{library}\NormalTok{(lubridate)}
\FunctionTok{library}\NormalTok{(knitr)}
\FunctionTok{library}\NormalTok{(kableExtra)}
\end{Highlighting}
\end{Shaded}

\begin{verbatim}

Attaching package: 'kableExtra'
\end{verbatim}

\begin{verbatim}
The following object is masked from 'package:dplyr':

    group_rows
\end{verbatim}

\begin{Shaded}
\begin{Highlighting}[]
\CommentTok{\# MAPE fonksiyonu}
\NormalTok{mape }\OtherTok{\textless{}{-}} \ControlFlowTok{function}\NormalTok{(actual, predicted) \{}
  \FunctionTok{mean}\NormalTok{(}\FunctionTok{abs}\NormalTok{((actual }\SpecialCharTok{{-}}\NormalTok{ predicted) }\SpecialCharTok{/}\NormalTok{ actual), }\AttributeTok{na.rm =} \ConstantTok{TRUE}\NormalTok{) }\SpecialCharTok{*} \DecValTok{100}
\NormalTok{\}}

\CommentTok{\# Tahminlerin eklendiği ve birleştirilmiş veri}
\NormalTok{combined\_data }\OtherTok{\textless{}{-}}\NormalTok{ epias\_daily }\SpecialCharTok{\%\textgreater{}\%}
  \FunctionTok{select}\NormalTok{(date, price,}
\NormalTok{         naturalgas, wind, lignite, darkcoal, importedcoal, fueloil,}
\NormalTok{         geothermal, dam, naphta, biomass, runofriver, other, demand, solar, natgasprice) }\SpecialCharTok{\%\textgreater{}\%}
  \FunctionTok{na.omit}\NormalTok{() }\SpecialCharTok{\%\textgreater{}\%}
  \FunctionTok{mutate}\NormalTok{(}\AttributeTok{pred\_model1 =} \FunctionTok{predict}\NormalTok{(model1, }\AttributeTok{newdata =}\NormalTok{ .)) }\SpecialCharTok{\%\textgreater{}\%}
  \FunctionTok{inner\_join}\NormalTok{(}
\NormalTok{    epias\_simplified\_daily }\SpecialCharTok{\%\textgreater{}\%}
      \FunctionTok{select}\NormalTok{(date,}
\NormalTok{             cheap\_thermal, naturalgas, importedcoal, renewables, demand, solar, dam, natgasprice) }\SpecialCharTok{\%\textgreater{}\%}
      \FunctionTok{na.omit}\NormalTok{() }\SpecialCharTok{\%\textgreater{}\%}
      \FunctionTok{mutate}\NormalTok{(}\AttributeTok{pred\_model2 =} \FunctionTok{predict}\NormalTok{(model, }\AttributeTok{newdata =}\NormalTok{ .)),}
    \AttributeTok{by =} \StringTok{"date"}
\NormalTok{  )}

\CommentTok{\# Aylık MAPE hesaplama}
\NormalTok{mape\_summary }\OtherTok{\textless{}{-}}\NormalTok{ combined\_data }\SpecialCharTok{\%\textgreater{}\%}
  \FunctionTok{mutate}\NormalTok{(}\AttributeTok{month =} \FunctionTok{floor\_date}\NormalTok{(date, }\StringTok{"month"}\NormalTok{)) }\SpecialCharTok{\%\textgreater{}\%}
  \FunctionTok{group\_by}\NormalTok{(month) }\SpecialCharTok{\%\textgreater{}\%}
  \FunctionTok{summarise}\NormalTok{(}
    \AttributeTok{MAPE\_Model1 =} \FunctionTok{mape}\NormalTok{(price, pred\_model1),}
    \AttributeTok{MAPE\_Model2 =} \FunctionTok{mape}\NormalTok{(price, pred\_model2)}
\NormalTok{  )}

\CommentTok{\# Ortalama MAPE satırını ekle}
\NormalTok{mape\_summary\_final }\OtherTok{\textless{}{-}} \FunctionTok{bind\_rows}\NormalTok{(}
\NormalTok{  mape\_summary,}
  \FunctionTok{summarise}\NormalTok{(mape\_summary,}
            \AttributeTok{month =} \FunctionTok{as.Date}\NormalTok{(}\StringTok{"9999{-}12{-}31"}\NormalTok{),}
            \AttributeTok{MAPE\_Model1 =} \FunctionTok{mean}\NormalTok{(MAPE\_Model1),}
            \AttributeTok{MAPE\_Model2 =} \FunctionTok{mean}\NormalTok{(MAPE\_Model2))}
\NormalTok{)}

\CommentTok{\# Tarih sütununu karakter formatına çevir}
\NormalTok{mape\_summary\_final }\OtherTok{\textless{}{-}}\NormalTok{ mape\_summary\_final }\SpecialCharTok{\%\textgreater{}\%}
  \FunctionTok{mutate}\NormalTok{(}\AttributeTok{month =} \FunctionTok{if\_else}\NormalTok{(month }\SpecialCharTok{==} \FunctionTok{as.Date}\NormalTok{(}\StringTok{"9999{-}12{-}31"}\NormalTok{), }\StringTok{"Average"}\NormalTok{, }\FunctionTok{format}\NormalTok{(month, }\StringTok{"\%Y{-}\%m"}\NormalTok{)))}

\CommentTok{\# Tabloyu yazdır}
\FunctionTok{kable}\NormalTok{(mape\_summary\_final, }\AttributeTok{digits =} \DecValTok{2}\NormalTok{, }\AttributeTok{caption =} \StringTok{"Monthly MAPE Comparison of Model 1 and Model 2"}\NormalTok{) }\SpecialCharTok{\%\textgreater{}\%}
  \FunctionTok{kable\_styling}\NormalTok{(}\AttributeTok{full\_width =} \ConstantTok{FALSE}\NormalTok{, }\AttributeTok{position =} \StringTok{"center"}\NormalTok{) }\SpecialCharTok{\%\textgreater{}\%}
  \FunctionTok{row\_spec}\NormalTok{(}\FunctionTok{nrow}\NormalTok{(mape\_summary\_final), }\AttributeTok{bold =} \ConstantTok{TRUE}\NormalTok{, }\AttributeTok{background =} \StringTok{"\#f2f2f2"}\NormalTok{, }
           \AttributeTok{extra\_css =} \StringTok{"border{-}top: 2px solid \#999;"}\NormalTok{)}
\end{Highlighting}
\end{Shaded}

\begin{longtable}[t]{lrr}
\caption{Monthly MAPE Comparison of Model 1 and Model 2}\\
\toprule
month & MAPE\_Model1 & MAPE\_Model2\\
\midrule
2023-01 & 3.11 & 2.71\\
2023-02 & 6.06 & 6.66\\
2023-03 & 7.05 & 5.51\\
2023-04 & 17.37 & 21.19\\
2023-05 & 9.31 & 11.19\\
\addlinespace
2023-06 & 9.52 & 11.38\\
2023-07 & 6.01 & 7.88\\
2023-08 & 4.85 & 6.78\\
2023-09 & 4.64 & 4.53\\
2023-10 & 4.29 & 3.17\\
\addlinespace
2023-11 & 6.27 & 7.84\\
2023-12 & 3.98 & 6.02\\
2024-01 & 6.74 & 9.20\\
2024-02 & 8.44 & 9.70\\
2024-03 & 5.82 & 5.49\\
\addlinespace
2024-04 & 17.65 & 22.54\\
2024-05 & 8.61 & 9.92\\
2024-06 & 8.79 & 10.63\\
2024-07 & 3.29 & 3.51\\
2024-08 & 3.06 & 3.86\\
\addlinespace
2024-09 & 3.89 & 3.94\\
2024-10 & 6.23 & 7.30\\
2024-11 & 4.62 & 4.58\\
2024-12 & 5.41 & 4.08\\
\cellcolor[HTML]{f2f2f2}{\textbf{Average}} & \cellcolor[HTML]{f2f2f2}{\textbf{6.88}} & \cellcolor[HTML]{f2f2f2}{\textbf{7.90}}\\
\bottomrule
\end{longtable}

MAPE, the most widely used performance metric in electricity price
forecasting, reveals an approximate 1\% difference in error between the
two models. The model without feature aggregation outperforms the
aggregated one, achieving the lowest error rate at 6.88\%.

In both of models, Natural Gas Price is the key indicator of prediction.

\section{4. Results and Key Takeaways}\label{results-and-key-takeaways}

This project focused on forecasting electricity prices in Turkey's
day-ahead market. The main objective was to demonstrate that accurate
price forecasting in this market does not necessarily require highly
complex models, and that successful results can be achieved with
relatively simple approaches. A forecasting model was developed using
publicly available data from EPİAŞ and BOTAŞ. Two different models were
built, with the only distinction being their approach to feature usage.
While one model included all features individually, the other applied a
feature aggregation strategy. Although the aggregated model was
initially expected to perform better, it delivered slightly worse
results. However, it provided clearer interpretability regarding the
influence of each feature on electricity prices.

The second model differed from the first by only about 1\% in terms of
error rate, which can be considered an acceptable margin. Given its
simplicity, the second model was initially assumed to be more efficient.
However, this outcome raises some important considerations. First, the
fact that both models underperformed compared to the first model
presents a challenge. Future improvements should focus on applying
feature aggregation or feature elimination techniques to achieve both a
simpler and more accurate model.

Additionally, some of the features used in the models exhibit high
correlation with each other, which may lead to redundancy. Reducing
multicollinearity and introducing alternative variables may help better
capture the influence of distinct factors on electricity prices.
Moreover, while this study relied on multiple linear regression, testing
other simple mathematical models could provide further insights into
prediction performance and model robustness.




\end{document}
